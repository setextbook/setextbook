\newcommand{\abstractionDef}{\textbf{abstraction}: Representation that is purposely missing details to focus attention on purpose of the object / idea / etc. being represented.}

\newcommand{\acceptanceCriterionDef}{\textbf{acceptance criterion}: A statement about functionality that, when satisfied, mean the functionality has been satisfactorily implemented.}

\newcommand{\agileDef}{\textbf{Agile}: A software process model and philosophy for managing and developing software projects. Agile values: Individuals and interactions, working software, customer collaboration, and responding to change.}

\newcommand{\businessCapabilityDef}{\textbf{business capability}: ``the potential of a business resource (or groups of resources) to produce customer value by acting on their environment via a process using other tangible and intangible resources'' \parencite{Michell2011AFA}}

\newcommand{\classDiagramDef}{\textbf{class diagram}: Visualization of how classes are built in relation to other classes in object-oriented software. Includes properties and methods of individual classes and ``has a'' and ``is a'' relationships between classes.}

\newcommand{\clientDef}{\textbf{client}\index{client}\index{customer} (a.k.a. customer): One or more people or organizations who are requesting the software be made and have decision-making authority about the software (e.g., because they are paying for it or otherwise providing resources).}

\newcommand{\clientServerArchitectureDef}{\textbf{client-server architecture}: Overall code design characterized by one component (the server) responding to requests and providing resources while other components (clients) request those resources.}

\newcommand{\codeDecayDef}{\textbf{code decay} (AKA software rot): Reduction of code quality over time. Can result in decreased maintainability, more bugs, and irretrievable failure.}

\newcommand{\codeSmellDef}{\textbf{code smell}: Aspect of code that indicates the code is of poor quality (e.g., has detriments to read- ability and maintainability).}

\newcommand{\componentDef}{\textbf{component}: Within a codebase, a unit of the code containing related functionality. Ideally, is both replaceable and reusable.}

\newcommand{\contingencyDef}{\textbf{contingency}: A future event or circumstance that may occur but depends on known and unknown factors. Can be difficult to predict far ahead of time.}

\newcommand{\couplingDef}{\textbf{coupling}: The degree to which one unit of code is dependent on another.}

\newcommand{\dodDef}{\textbf{Definition of Done (DoD)}: The set of acceptance criteria which, once satisfied, mean a user story has been satisfactorily implemented.}

\newcommand{\eisenhowerMatrixDef}{\textbf{Eisenhower matrix}: 2x2 grid for helping decide whether to do, delegate, schedule, or eliminate a task based on its urgency and importance.}

\newcommand{\encapsulationDef}{\textbf{encapsulation}: In object-oriented programming, (1) combining data and the methods that act upon that data into one unit of code or (2) preventing external direct access to data within a unit of code.}

\newcommand{\estimationDef}{\textbf{estimation}: Figuring out ahead of time how long a task is likely to take.}

\newcommand{\eventualConsistencyDef}{\textbf{eventual consistency}: Characteristic of software systems where different parts of the system can have less up-to-date information (e.g., state, data) than other parts but the inconsistencies are temporary.}

\newcommand{\extensibleDef}{\textbf{extensible}: Built in such a way to support adding more functionality later.}

\newcommand{\extremeProgrammingDef}{\textbf{Extreme Programming (XP)}: Agile framework that prioritizing customer satisfaction and communication, short development cycles, iteration, frequent releases, code review, teamwork, pair programming, required unit testing, and only implementing functionality that's needed.}

\newcommand{\fistOfFiveDef}{\textbf{fist of five}: A method for gauging and building group consensus that uses a 6-level voting system (zero to five fingers).}

\newcommand{\focusGroupDef}{\textbf{focus group} (in usability engineering): A moderated discussion between researcher and a small number of potential users (usually 6-12) during which the researcher tries to gather information about the participants' attitudes, opinions, motivations, concerns, and problems related to a specific product or topic.\parencite{odimegwu00}}

\newcommand{\functionalRequirementDef}{\textbf{functional requirement}: Description of what functionality the software needs to have.}

\newcommand{\ganttChartDef}{\textbf{Gantt chart}: Horizontal bar chart showing start and end times of activities within a project schedule, along a timeline.}

\newcommand{\guiDef}{\textbf{graphical user interface (GUI)}: A user interface with interactive graphics, in contrast to a text-based user interface.}

\newcommand{\groundRulesDef}{\textbf{ground rules}: A set of statements about the team, agreed to by each team member, for avoiding team conflict and dysfunction.}

\newcommand{\highFidelityPrototypeDef}{\textbf{high-fidelity prototype}: A polished illustration that looks like a finished, publishable user interface design (especially a GUI). Almost always digital.}

\newcommand{\highLevelArchitectureDef}{\textbf{high-level architecture}: Abstract representation of overall code design; covers all parts of the software.}

\newcommand{\ideDef}{\textbf{IDE}: Integrated development environment. Software specifically for creating software.}

\newcommand{\idealDaysDef}{\textbf{ideal days}: The number of days it would take to complete the work if the work could be 100\% focused on.}

\newcommand{\incrementDef}{\textbf{increment}: In software, a measurable increase in functionality.}

\newcommand{\interactionDesignDef}{\textbf{interaction design}: An approach to technology design that involves helping users understand what's happening with the technology, what just happened, and what they can do \parencite{norman13}.}

\newcommand{\interactionDiagramDef}{\textbf{interaction diagram}: Visualization of collaboration between different parts of software.}

\newcommand{\investDef}{\textbf{INVEST}: Characteristics of good user stories (independent, negotiable, valuable, estimable, small, testable) \parencite{wake}.}

\newcommand{\iterationDef}{\textbf{iteration}: Verb: Revision. Noun (in Agile): A time-boxed software development cycle.}

\newcommand{\iterationPlanDef}{\textbf{iteration plan}: In Agile, establishing what will be done during a development cycle.}

\newcommand{\lowFidelityPrototypeDef}{\textbf{low-fidelity prototype}: A \textbf{rough} sketch of a user interface design (especially a GUI). Can be hand-drawn or digital.}

\newcommand{\maintenanceDef}{\textbf{maintenance}: Development activities that improve software but that are unrelated to implementing new features (e.g., correcting bugs, improving organization of code, etc.).}

\newcommand{\msmDef}{\textbf{managerial skill mix (MSM)}: Three categories of skills used by managers: (1) interpersonal, (2) technical, (3) administrative/conceptual.}

\newcommand{\mediumFidelityPrototypeDef}{\textbf{medium-fidelity prototype}: A careful and detailed illustration of a user interface design (especially a GUI). Can be hand-drawn, but digital is more common.}

\newcommand{\methodDef}{\textbf{method}: A pre-established way of achieving a specific outcome.}

\newcommand{\microservicesArchitectureDef}{\textbf{microservices architecture}: Overall code design characterized by multiple independent components that each run in their own process and communicate between one another without direct access.}

\newcommand{\mitigationPlanDef}{\textbf{mitigation plan}: What you will do if a contingency happens.}

\newcommand{\monolithArchitectureDef}{\textbf{monolith architecture}: Overall code design characterized by being in one or few pieces; cannot be easily divided into components that run separately and are independently useful.}

\newcommand{\mvpDef}{\textbf{minimum viable product (MVP)}: A low-effort or low-expense effort that results in you being able to better estimate whether people will want to use your product---before the product is fully developed.\parencite{olsen15}}

\newcommand{\nonFunctionalRequirementDef}{\textbf{non-functional requirement}: Description of how well software is expected to perform.}

\newcommand{\paperPrototypeDef}{\textbf{paper prototype}: A hand-drawn sketch used to communicate a potential user interface design to be implemented, especially a graphical user interface design \parencite{snyder03}.}

\newcommand{\planningPokerDef}{\textbf{planning poker}: In Agile, a consensus-based method of assigning estimates to a task that involves individuals on a team each making their own estimate privately, then sharing with the team, discussing, and re-estimating as needed.}

\newcommand{\productBacklogDef}{\textbf{Product Backlog}: In Agile Scrum, an ordered list of all that is known to be needed to improve a product.}

\newcommand{\projectManagementDef}{\textbf{project management}: The process of planning and executing a project while balancing the time, cost, and scope constraints.}

\newcommand{\projectManagementSystemDef}{\textbf{project management system}: Software for planning, organizing, and otherwise carrying out a project.}

\newcommand{\projectNetworkDef}{\textbf{project network}: Graph showing the order in which a project's activities are to be completed.}

\newcommand{\projectPriorityMatrixDef}{\textbf{project priority matrix:} 3x3 grid for documenting how to respond when there are potential changes to a project's time, cost, or scope. Options: Only positive change allowed (constrain), negative change allowed (accept), or positive change sought (enhance).}

\newcommand{\qualityAttributeDef}{\textbf{quality attribute}: A characteristic of software used to describe how good it is.}

\newcommand{\raciMatrixDef}{\textbf{RACI matrix}: In project management, a chart for defining which roles are responsible (R) and accountable (A) for a task or deliverable and which roles should be consulted (C) or informed (I) about the status of the task or deliverable.}

\newcommand{\refactoringDef}{\textbf{refactoring}: Improving code design without changing what the code does.}

\newcommand{\releasePlanDef}{\textbf{release plan}: What will be completed for a specific software release and when the release will occur.}

\newcommand{\requirementDef}{\textbf{requirement}: A rule the software must conform to: What the software must to, how well it must do what it does, or the software's limitations or constraints.}

\newcommand{\requirementsElicitationDef}{\textbf{requirements elicitation}: The process of gathering requirements from project stakeholders.}

\newcommand{\requirementsSpecificationDef}{\textbf{requirements specification}: Converting stakeholder requests into written requirements.}

\newcommand{\riskDef}{\textbf{risk}: Estimated probability of a negative contingency given known and unknown factors.}

\newcommand{\sequenceDiagramDef}{\textbf{sequence diagram}: Interaction diagram showing how different participants (e.g., users, software components, classes, etc.) collaborate during a single use case.}

\newcommand{\serviceDef}{\textbf{service}: A unit of software that received and fulfills requests.}

\newcommand{\schedulingDef}{\textbf{scheduling}: Deciding when project activities are to be completed, how long they will take, and what resources are needed to complete them.}

\newcommand{\scrumDef}{\textbf{Scrum}: An Agile framework ``for developing and sustaining complex products.'' \parencite{schwaber2020scrum}}

\newcommand{\sdlcDef}{\textbf{software development lifecycle (SDLC)}: Phases through which a software's development proceeds: requirements, design, implementation, testing, maintenance.}

\newcommand{\softwareArchitectureDef}{\textbf{software architecture}: Code design. Can be shown at different levels of abstraction and detail.}

\newcommand{\softwareEngineeringDef}{\textbf{software engineering}: The art and science of using different methods to efficiently create extensible, sustainable programs that solve problems people care about.}

\newcommand{\softwareProcessModelDef}{\textbf{software process model}: A philosophy and/or set of approaches for software development and/or software project management.}

\newcommand{\spikeDef}{\textbf{spike}: A quick and to-the-point investigation for gathering information to help the team answer a question or choose a development path.}

\newcommand{\sprintBacklogDef}{\textbf{Sprint Backlog}: In Scrum, the set of activities to be completed during a Sprint (from Product Backlog), the associated Sprint Goal, and a plan for completing the activities.}

\newcommand{\srsDef}{\textbf{Software Requirements Specification (SRS)}: A document that contains software requirements.}

\newcommand{\stakeholderDef}{\textbf{stakeholder}: Anyone who is or will be affected by the software or its development (e.g., clients, companies, users, developers, managers, politicians, etc.)}

\newcommand{\storyPointsDef}{\textbf{story points}: A method for estimating an activity based on its size relative to other activities. Scale established by team.}

\newcommand{\sustainabilityDef}{\textbf{sustainability}: Degree to which software can continue to function over time (e.g., measured in time and how well the software is functioning).}

\newcommand{\taskManagementSystemDef}{\textbf{task management system}: Software for planning and organizing project activities.}

\newcommand{\technicalDebtDef}{\textbf{technical debt}: Time and resources you (or someone else) will need to spend on modifying your software in the future because of the poor decisions you're making in the present.}

\newcommand{\techStackDef}{\textbf{tech stack}: The set of programming languages, frameworks, and other technologies chosen or needed for implementing a piece of software.}

\newcommand{\thinkAloudProtocolDef}{\textbf{think-aloud protocol}: A method for gathering feedback about the usability of a design that involves a test user speaking their thoughts as they interact with the design \parencite{lewis93}. More information: \url{https://tinyurl.com/think-aloud-protocol}}

\newcommand{\tripleConstraintDef}{\textbf{triple constraint}: In project management, the three limiting factors that govern project execution: time, cost, and scope. Scope includes quality. Cost includes spending money and resources.}

\newcommand{\tuckmanDef}{\textbf{Tuckman's model of team development}: A five-stage model of how a team develops over time: (1) forming, (2) storming, (3) norming, (4) performing, (5) adjourning.}

\newcommand{\uatDef}{\textbf{user acceptance testing (UAT)}: Formally testing software with end-users to check not only whether it performs as expected but also whether end-users will use it. Typically performed before the software is released.}

\newcommand{\umlDef}{\textbf{UML}: Unified modeling language: A set of notation and methods for describing and designing software.}

\newcommand{\usabilityTestingDef}{\textbf{usability testing}: Observing people while they try to use your software.\parencite{barnum20}}

\newcommand{\userInterfaceDef}{\textbf{user interface (UI)}: What a user interacts with to operate a system (e.g., a graphical user interface, a command-line interface, a virtual or augmented reality interface, etc.).}

\newcommand{\useCaseDef}{\textbf{use case}\index{use case}: ``A contract for the behavior of the system under discussion'' \parencite{cockburn01}}

\newcommand{\userStoryDef}{\textbf{user story}: ``Short, simple descriptions of a feature told from the perspective of the person who desires the new capability, usually a user or customer of the system.'' \parencite{cohn}}

\newcommand{\validationDef}{\textbf{validation}: Confirming that software meets users' needs (``did we build the right software?'').}

\newcommand{\velocityDef}{\textbf{velocity}: In Agile, a measure of how much work is being completed.}

\newcommand{\verificationDef}{\textbf{verification}: Confirming that software satisfied its requirements (``did we build the software right?'').}

\newcommand{\waterfallDef}{\textbf{waterfall} (software process model): Way of going about software development and management that is characterized by extensive planning, comprehensive documentation, and moving linearly through stages of the software development lifecycle (SDLC).}